% Impala Operating System
%
% Copyright (C) 2009 University of Wroclaw. Department of Computer Science
%    http://www.ii.uni.wroc.pl/
% Copyright (C) 2009 Mateusz Kocielski, Artur Koninski, Pawel Wieczorek
%    http://trzask.codepainters.com/impala/trac/
% All rights reserved.
%
% Redistribution and use in source and binary forms, with or without
% modification, are permitted provided that the following conditions
% are met:
% 1. Redistributions of source code must retain the above copyright
%  notice, this list of conditions and the following disclaimer.
% 2. Redistributions in binary form must reproduce the above copyright
%  notice, this list of conditions and the following disclaimer in the
%  documentation and/or other materials provided with the distribution.
%
% THIS SOFTWARE IS PROVIDED BY AUTHOR AND CONTRIBUTORS ``AS IS'' AND
% ANY EXPRESS OR IMPLIED WARRANTIES, INCLUDING, BUT NOT LIMITED TO, THE
% IMPLIED WARRANTIES OF MERCHANTABILITY AND FITNESS FOR A PARTICULAR PURPOSE
% ARE DISCLAIMED.  IN NO EVENT SHALL AUTHOR OR CONTRIBUTORS BE LIABLE
% FOR ANY DIRECT, INDIRECT, INCIDENTAL, SPECIAL, EXEMPLARY, OR CONSEQUENTIAL
% DAMAGES (INCLUDING, BUT NOT LIMITED TO, PROCUREMENT OF SUBSTITUTE GOODS
% OR SERVICES; LOSS OF USE, DATA, OR PROFITS; OR BUSINESS INTERRUPTION)
% HOWEVER CAUSED AND ON ANY THEORY OF LIABILITY, WHETHER IN CONTRACT, STRICT
% LIABILITY, OR TORT (INCLUDING NEGLIGENCE OR OTHERWISE) ARISING IN ANY WAY
% OUT OF THE USE OF THIS SOFTWARE, EVEN IF ADVISED OF THE POSSIBILITY OF
% SUCH DAMAGE.
%
% $Id$

\section{Uruchomienie systemu.}

Dyskietka z~naszym systemem ma zainstalowany program �aduj�cy GRUB, kt�ry mo�e
zosta� uruchomiony przez BIOS. Je�eli w~komputerze ju� istnieje program �aduj�cy
obs�uguj�cy format ELF (\ref{tech:elf}) oraz kompresj� programu \texttt{gzip}
to~mo�na sprawdzi� czy jest w~stanie r�cznie uruchomi� nasz system, bez 
konfiguracji BIOSu aby uruchamia� stacj� dyskietek.
Dla przyk�adu, je�eli ju� u�ywamy programu GRUB na dysku twardym,
to mo�na dopisa� nast�puj�ce linijki do jego konfiguracji:

\begin{verbatim}
# Wpis 1
title Uruchom system ze stacji dyskietek A:
    root (fd0)
    chainloader +1

# Wpis 2
title System operacyjny Impala-LPP
    root (fd0)
    kernel /boot/impala.gz
\end{verbatim}

Pierwszy wpis uruchamia program �aduj�cy z~dyskietki, tak jak by to~zrobi�
BIOS, a~drugi wpis bezpo�rednio uruchamia j�dro naszego systemu.

Po za�adowaniu j�dra nast�pi proces inicjalizacji systemu, w~sk�ad kt�rego
wchodzi rozpakowanie danych systemu z~dyskietki oraz uruchomienie
skrypt�w startowych. Po wykonaniu tej fazy zostan� udost�pnione trzy
wirtualne terminale, na kt�rych b�dzie uruchomiona pow�oka.

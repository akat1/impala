% Impala Operating System
%
% Copyright (C) 2009 University of Wroclaw. Department of Computer Science
%    http://www.ii.uni.wroc.pl/
% Copyright (C) 2009 Mateusz Kocielski, Artur Koninski, Pawel Wieczorek
%    http://trzask.codepainters.com/impala/trac/
% All rights reserved.
%
% Redistribution and use in source and binary forms, with or without
% modification, are permitted provided that the following conditions
% are met:
% 1. Redistributions of source code must retain the above copyright
%  notice, this list of conditions and the following disclaimer.
% 2. Redistributions in binary form must reproduce the above copyright
%  notice, this list of conditions and the following disclaimer in the
%  documentation and/or other materials provided with the distribution.
%
% THIS SOFTWARE IS PROVIDED BY AUTHOR AND CONTRIBUTORS ``AS IS'' AND
% ANY EXPRESS OR IMPLIED WARRANTIES, INCLUDING, BUT NOT LIMITED TO, THE
% IMPLIED WARRANTIES OF MERCHANTABILITY AND FITNESS FOR A PARTICULAR PURPOSE
% ARE DISCLAIMED.  IN NO EVENT SHALL AUTHOR OR CONTRIBUTORS BE LIABLE
% FOR ANY DIRECT, INDIRECT, INCIDENTAL, SPECIAL, EXEMPLARY, OR CONSEQUENTIAL
% DAMAGES (INCLUDING, BUT NOT LIMITED TO, PROCUREMENT OF SUBSTITUTE GOODS
% OR SERVICES; LOSS OF USE, DATA, OR PROFITS; OR BUSINESS INTERRUPTION)
% HOWEVER CAUSED AND ON ANY THEORY OF LIABILITY, WHETHER IN CONTRACT, STRICT
% LIABILITY, OR TORT (INCLUDING NEGLIGENCE OR OTHERWISE) ARISING IN ANY WAY
% OUT OF THE USE OF THIS SOFTWARE, EVEN IF ADVISED OF THE POSSIBILITY OF
% SUCH DAMAGE.
%
% $Id$

\section{Pow�oka systemowa.}

Do~naszego systemu operacyjnego jest do��czona pow�oka stworzona przez
Alqmuista Kennetha w~1993 dla systemu BSD, znana jako \emph{Almquist Shell}
lub~\emph{ASH}.

Ta dokumentacja nie zawiera dok�adnego opisu wbudowanych polece� tej pow�oki 
oraz j�zyka jakim si� pos�uguje, te~opisy znajduj� si� w~\cite{freebsd:sh} oraz 
\cite{posix:sh}. Poni�ej znajduje si� jedynie opis podstawowych komend niezb�dnych
do swobodnego u�ytkownia systemu.

\begin{itemize}
\item \texttt{cd} - zmiana aktualnego katalogu.
\item \texttt{echo} - wy�wietlenie dostarczonego napisu.
\item \texttt{exit} - zako�czenie pracy.
\item \texttt{pwd} - wy�wietlenie aktualnego katalogu.
\end{itemize}

% Impala Operating System
%
% Copyright (C) 2009 University of Wroclaw. Department of Computer Science
%    http://www.ii.uni.wroc.pl/
% Copyright (C) 2009 Mateusz Kocielski, Artur Koninski, Pawel Wieczorek
%    http://trzask.codepainters.com/impala/trac/
% All rights reserved.
%
% Redistribution and use in source and binary forms, with or without
% modification, are permitted provided that the following conditions
% are met:
% 1. Redistributions of source code must retain the above copyright
%  notice, this list of conditions and the following disclaimer.
% 2. Redistributions in binary form must reproduce the above copyright
%  notice, this list of conditions and the following disclaimer in the
%  documentation and/or other materials provided with the distribution.
%
% THIS SOFTWARE IS PROVIDED BY AUTHOR AND CONTRIBUTORS ``AS IS'' AND
% ANY EXPRESS OR IMPLIED WARRANTIES, INCLUDING, BUT NOT LIMITED TO, THE
% IMPLIED WARRANTIES OF MERCHANTABILITY AND FITNESS FOR A PARTICULAR PURPOSE
% ARE DISCLAIMED.  IN NO EVENT SHALL AUTHOR OR CONTRIBUTORS BE LIABLE
% FOR ANY DIRECT, INDIRECT, INCIDENTAL, SPECIAL, EXEMPLARY, OR CONSEQUENTIAL
% DAMAGES (INCLUDING, BUT NOT LIMITED TO, PROCUREMENT OF SUBSTITUTE GOODS
% OR SERVICES; LOSS OF USE, DATA, OR PROFITS; OR BUSINESS INTERRUPTION)
% HOWEVER CAUSED AND ON ANY THEORY OF LIABILITY, WHETHER IN CONTRACT, STRICT
% LIABILITY, OR TORT (INCLUDING NEGLIGENCE OR OTHERWISE) ARISING IN ANY WAY
% OUT OF THE USE OF THIS SOFTWARE, EVEN IF ADVISED OF THE POSSIBILITY OF
% SUCH DAMAGE.
%
% $Id$

\section{Szczeg�y techniczne.}

\subsection{Obraz j�dra.}


\subsection{Obrazy program�w.}


\subsection{Rozruch systemu.}

Skompresowany obraz j�dra (\texttt{/boot/impala.gz}) jest �adowany przez
program GRUB, zainstalowany w~sektorze rozruchowym dyskietki. 


Z~dzia�ania j�dra w~wysokich adresach (\ref{vm:mmap}) wynika trudno��
techniczna przy �adawaniu systemu. Program �aduj�cy nie przygotowuje
mechanizmu stronicowania (pami�ci wirtualnej). W zwi�zku z czym og�lny
kod j�dra jest bezu�yteczny. 
Poniewa� format ELF umo�liwia tworzenie wielu sekcji posiadaj�cych r�ne
atrybuty to problem zosta� rozwi�zany przez wprowadzenie
specjalnej sekcji z kodem \texttt{.bootstrap}. Sekcja w~odr�nieniu
od standardowej sekcji z~kodem \texttt{.text} jest przystosowana
do pracy w niskich adresach, w tych w jakie jest �adowane j�dro.

Kod rozruchowy z~tej sekcji konfiguruje mechanizm stronicowania aby
odwzorowa� kod j�dra w~wysokich adresach. Po tej czynno�ci jest 
uruchamiana niskopoziomowa inicjacja systemu, a w niej przygotowanie
modu�u pami�ci wirtualnej, przerwa�, czasomierza i karty graficznej.

...

\subsection{System budowania.}



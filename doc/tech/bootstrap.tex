% Impala Operating System
%
% Copyright (C) 2009 University of Wroclaw. Department of Computer Science
%    http://www.ii.uni.wroc.pl/
% Copyright (C) 2009 Mateusz Kocielski, Artur Koninski, Pawel Wieczorek
%    http://trzask.codepainters.com/impala/trac/
% All rights reserved.
%
% Redistribution and use in source and binary forms, with or without
% modification, are permitted provided that the following conditions
% are met:
% 1. Redistributions of source code must retain the above copyright
%  notice, this list of conditions and the following disclaimer.
% 2. Redistributions in binary form must reproduce the above copyright
%  notice, this list of conditions and the following disclaimer in the
%  documentation and/or other materials provided with the distribution.
%
% THIS SOFTWARE IS PROVIDED BY AUTHOR AND CONTRIBUTORS ``AS IS'' AND
% ANY EXPRESS OR IMPLIED WARRANTIES, INCLUDING, BUT NOT LIMITED TO, THE
% IMPLIED WARRANTIES OF MERCHANTABILITY AND FITNESS FOR A PARTICULAR PURPOSE
% ARE DISCLAIMED.  IN NO EVENT SHALL AUTHOR OR CONTRIBUTORS BE LIABLE
% FOR ANY DIRECT, INDIRECT, INCIDENTAL, SPECIAL, EXEMPLARY, OR CONSEQUENTIAL
% DAMAGES (INCLUDING, BUT NOT LIMITED TO, PROCUREMENT OF SUBSTITUTE GOODS
% OR SERVICES; LOSS OF USE, DATA, OR PROFITS; OR BUSINESS INTERRUPTION)
% HOWEVER CAUSED AND ON ANY THEORY OF LIABILITY, WHETHER IN CONTRACT, STRICT
% LIABILITY, OR TORT (INCLUDING NEGLIGENCE OR OTHERWISE) ARISING IN ANY WAY
% OUT OF THE USE OF THIS SOFTWARE, EVEN IF ADVISED OF THE POSSIBILITY OF
% SUCH DAMAGE.
%
% $Id$

\section{Rozruch systemu}

% ten dzia� jest zama�y, trzeba go wrzuci� gdzie� indziej :)

J�dro systemu operacyjnego jest �adowane przez program \emph{GRUB}. Plik j�dra
zawiera w sobie informacje dla programu �aduj�cego w jaki adres nale�y je odczyta�
oraz gdzie znajduje si� procedura uruchamiaj�ca. 

Kod odpowiedzialny za rozruch jest umieszczony w sekcji \texttt{.bootstrap} w pliku
jadra. Jego zadanie to uruchomi� podstawow� obs�ug� sprz�tu (\texttt{init\_x86()}),
a nast�pnie uruchami� g��wn� procedur� (\texttt{kmain()}).

G��wna procedura j�dra inicjalizuje pokolei podsystemy oraz uruchamia pierwszy program
u�ytkownika \texttt{/sbin/init}.


